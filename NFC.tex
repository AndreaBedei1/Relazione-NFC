\documentclass[a4paper,11pt]{report}
\usepackage[T1]{fontenc}
\usepackage[italian]{babel}

\title{\Huge RELAZIONE NFC}
\author{Andrea Bedei}
\date{7 Marzo 2023}

\setcounter{tocdepth}{4}
\begin{document}
\maketitle
\tableofcontents

\chapter{Introduzione}
\section{Significato del termine NFC}
L'acronimo NFC significa "Near Field Communication" che in italiano significa "Comunicazione a Campo Vicino". \\NFC è una tecnologia di comunicazione wireless a corto raggio che consente il trasferimento di dati tra dispositivi vicini (generalmente entro un raggio di pochi centimetri) senza la necessità di un collegamento fisico. La tecnologia NFC si basa sulle onde radio a bassa frequenza (13,56 MHz) e utilizza la modulazione dell'ampiezza per trasmettere dati.\\
In generale, NFC è una tecnologia molto versatile che offre numerose possibilità di utilizzo per semplificare la vita quotidiana delle persone.

\chapter{Come funziona NFC}
I principi fondamentali di NFC (Near Field Communication) si basano sulla trasmissione di dati attraverso onde elettromagnetiche a corto raggio. Questa tecnologia consente a due dispositivi di comunicare tra loro a una distanza molto breve (generalmente entro un raggio di pochi centimetri) senza la necessità di un collegamento fisico.\\\\ La trasmissione dati avviene attraverso onde radio a bassa frequenza (13,56 MHz) che permettono una comunicazione sicura e veloce tra i dispositivi. La trasmissione dati avviene utilizzando la modulazione di ampiezza, ovvero la variazione dell'intensità del segnale elettromagnetico in modo da rappresentare i dati trasmessi.\\\\ Per utilizzare NFC è necessario disporre di alcuni componenti hardware fondamentali, ovvero l'antenna e il chip NFC. L'antenna è un dispositivo che permette di trasmettere e ricevere segnali elettromagnetici. Il chip NFC, invece, è il componente che permette di elaborare i dati trasmessi e di gestire la comunicazione NFC.\\\\Esistono due modalità di comunicazione NFC: la modalità attiva e la modalità passiva. Nella modalità attiva, il dispositivo trasmette attivamente i dati attraverso l'antenna NFC. In questo caso, il dispositivo si comporta come un emittente di onde radio e l'altro dispositivo si comporta come un ricevitore.\\\\ Nella modalità passiva, invece, il dispositivo riceve passivamente i dati trasmessi dall'altro dispositivo. In questo caso, l'altro dispositivo funge da emittente e il dispositivo NFC funge da ricevitore.\\\\ In entrambe le modalità di comunicazione, il processo di trasmissione dei dati è veloce e sicuro, poiché la distanza di comunicazione è molto breve e il segnale elettromagnetico viene trasmesso in modo criptato. Questo rende NFC una tecnologia molto utile per molti usi diversi, tra cui i pagamenti mobili, il trasferimento di file, il controllo degli accessi e molte altre applicazioni.

\chapter{Utilizzi di NFC}
I principali casi d'uso sono innumerevoli, tra i principali abbiamo:
\begin{enumerate}
    \item \textbf{Pagamenti mobili}: NFC è ampiamente utilizzato per i pagamenti mobili, grazie alla sua sicurezza e facilità d'uso. I dispositivi abilitati per NFC, come smartphone e smartwatch, possono essere utilizzati per effettuare pagamenti senza contatto in negozi fisici, ristoranti, distributori automatici e altri luoghi.
    \item \textbf{Condivisione di file}: NFC consente il trasferimento rapido e facile di file come immagini, video, musica e documenti tra dispositivi compatibili. I dispositivi possono essere posizionati uno accanto all'altro per avviare la comunicazione NFC e scambiare i dati.
    \item \textbf{Accesso elettronico}: NFC è utilizzato per il controllo degli accessi, ad esempio nei parcheggi, negli uffici e negli edifici pubblici. I dispositivi abilitati per NFC, come tessere e smartphone, possono essere utilizzati per accedere a luoghi protetti, senza la necessità di utilizzare una chiave fisica o una carta magnetica.
    \item \textbf{Smart home}: NFC può essere utilizzato per controllare i dispositivi smart home come luci, termostati e sistemi di sicurezza. Ad esempio, è possibile programmare una chiave NFC per attivare o disattivare il sistema di sicurezza di casa.
    \item \textbf{Biglietti elettronici}: NFC è ampiamente utilizzato per i biglietti elettronici per il trasporto pubblico, gli eventi sportivi e i concerti. I dispositivi abilitati per NFC, come smartphone e smartwatch, possono essere utilizzati come biglietti virtuali, che possono essere scansionati o letti da altri dispositivi compatibili.
    \item \textbf{Etichette intelligenti}: NFC può essere utilizzato per creare etichette intelligenti per prodotti, come ad esempio i codici QR. Le etichette NFC possono essere scansionate con un dispositivo abilitato per NFC per accedere a informazioni sul prodotto, come ad esempio il prezzo, la descrizione e le recensioni dei clienti.
    \item \textbf{Assistenza sanitaria}: NFC può essere utilizzato nel settore sanitario per tracciare i pazienti e gli attrezzi medici. Ad esempio, i pazienti possono indossare braccialetti o schede NFC per identificarsi e accedere ai loro dati sanitari, mentre gli attrezzi medici possono essere etichettati con tag NFC per tracciare il loro utilizzo e manutenzione.
    \item \textbf{Marketing}: NFC può essere utilizzato per creare campagne di marketing interattive e coinvolgenti. Ad esempio, i cartelloni pubblicitari possono essere dotati di tag NFC per consentire ai clienti di accedere a promozioni esclusive, video e altre informazioni sul prodotto.
    \item \textbf{Tracciamento delle attività fisiche}: NFC può essere utilizzato per tracciare le attività fisiche degli utenti, come ad esempio la corsa e il ciclismo. Ad esempio, gli utenti possono indossare braccialetti o orologi abilitati per NFC per registrare i loro dati di attività fisica e sincronizzarli con le app per il fitness.
    \item \textbf{Identificazione dei prodotti}: NFC può essere utilizzato per identificare i prodotti e verificare la loro autenticità. Ad esempio, i produttori di prodotti di lusso possono utilizzare tag NFC per proteggere i loro prodotti contro la contraffazione e garantire ai clienti l'autenticità dei loro acquisti.
\end{enumerate}

\chapter{Novità sugli utilizzi futuri}
\begin{enumerate}
    \item \textbf{Smart Home}: La casa intelligente è un sistema che consente di controllare e automatizzare le attività domestiche utilizzando la tecnologia. L'utilizzo della tecnologia NFC e RFID consente di personalizzare e programmare un'applicazione specifica utilizzando il proprio smartphone. Ad esempio, posizionando un tag NFC su una lampada, l'utente può programmare il tag in modo che, una volta che viene letto dallo smartphone, accenda o spegni la lampada. Allo stesso modo, un tag NFC può essere posizionato vicino alla porta di casa, in modo che quando l'utente avvicina il proprio smartphone al tag, si sblocca la porta. In questo modo, la tecnologia NFC consente di creare una casa intelligente e personalizzata, in cui le attività domestiche possono essere controllate e automatizzate in modo semplice e intuitivo.
    \item Internet of things (IoT) e 5G: L'IoT è un sistema in cui i dispositivi sono interconnessi tramite Internet, consentendo loro di comunicare tra loro e di scambiare dati. La tecnologia NFC è un componente chiave per la realizzazione di soluzioni IoT, poiché consente la comunicazione senza fili a breve distanza tra i dispositivi. Inoltre, la tecnologia 5G offre una velocità di trasmissione dei dati molto elevata e una latenza ridotta, rendendola particolarmente adatta per applicazioni IoT. Ciò significa che l'utilizzo di dispositivi NFC diventerà sempre più importante per implementare le soluzioni IoT e garantire la loro efficienza.
    \item \textbf{Applicazioni integrate per smartphone}: La tecnologia NFC consente di creare tag intelligenti che possono essere utilizzati per configurare applicazioni per smartphone. Ad esempio, un'azienda può utilizzare un tag NFC per creare un programma di fedeltà. Quando un cliente avvicina il proprio smartphone al tag NFC, riceve automaticamente punti nel proprio programma di fedeltà. Allo stesso modo, i tag NFC possono essere utilizzati per controllare l'accesso a determinati servizi, come l'accesso come membro o l'ingresso in un'area restritta. Inoltre, la tecnologia NFC può essere utilizzata per semplificare le attività quotidiane come pagamenti senza contatto, l'avvio di app o la connessione ad un'apertura Wi-Fi. In generale, l'utilizzo della tecnologia NFC consente di creare soluzioni personalizzate e integrate per smartphone, semplificando l'interazione tra l'utente e il mondo che lo circonda.
    \item \textbf{Identificazione personale}: NFC può essere utilizzato per creare sistemi di identificazione personale altamente sicuri e affidabili. Ad esempio, i passaporti biometrici potrebbero utilizzare la tecnologia NFC per garantire l'identità dei viaggiatori e semplificare il processo di controllo dei confini.
    \item \textbf{Intelligenza artificiale (AI)}: NFC potrebbe essere utilizzato per creare nuove applicazioni di intelligenza artificiale, come ad esempio i sistemi di riconoscimento vocale e facciale. Ad esempio, gli utenti potrebbero utilizzare i loro dispositivi NFC per accedere a servizi di assistenza vocale o per sbloccare i loro dispositivi con il riconoscimento facciale.
    \item \textbf{Sicurezza}: NFC potrebbe essere utilizzato per creare sistemi di sicurezza altamente sofisticati, come ad esempio le impronte digitali e i sensori biometrici. Ad esempio, le banche potrebbero utilizzare la tecnologia NFC per garantire che le transazioni siano sicure e autentiche.
    \item \textbf{Smart packaging}: NFC può essere utilizzato per creare smart packaging, ovvero confezioni che interagiscono con i dispositivi degli utenti. Ad esempio, i produttori di alimenti potrebbero utilizzare la tecnologia NFC per fornire informazioni nutrizionali o ricette suggerite ai clienti.
    \item \textbf{Salute e benessere}: NFC può essere utilizzato per creare soluzioni di monitoraggio della salute e del benessere. Ad esempio, gli utenti potrebbero utilizzare dispositivi NFC per monitorare i loro parametri vitali o per accedere a servizi di assistenza sanitaria.
\end{enumerate}

\section{Chip sotto pelle}
L'implantologia di chip NFC sotto pelle è un argomento che ha suscitato un certo interesse e dibattito negli ultimi anni. Questa tecnologia può offrire una varietà di benefici, come l'accesso senza contatto a edifici o veicoli, l'identificazione personale o il pagamento senza contatto. Tuttavia, ci sono anche preoccupazioni legate alla privacy, alla sicurezza e alla salute.\\\\ Per quanto riguarda la privacy, l'uso di chip NFC sotto pelle potrebbe consentire la tracciabilità degli individui e delle loro attività, sia da parte di aziende che di governi. Ci sono preoccupazioni anche riguardo la sicurezza dei dati memorizzati nei chip, poiché potrebbero essere vulnerabili ad attacchi informatici o a furto di identità. Inoltre, ci sono preoccupazioni sulla salute, come la possibilità che il corpo rifiuti il chip o che si verifichi un'infezione.\\\\ Inoltre, l'implantologia di chip NFC sotto pelle richiede una procedura chirurgica, anche se relativamente semplice, che deve essere eseguita da personale sanitario specializzato in un ambiente sterile. Ciò aumenta il rischio di infezioni, sanguinamento e altri effetti collaterali, sebbene la maggior parte delle procedure di impianto dei chip NFC sotto pelle siano generalmente sicure e prive di complicazioni.\\\\ Infine, l'implantologia di chip NFC sotto pelle è ancora una tecnologia emergente e non è stata ancora completamente sperimentata e regolamentata. Ad esempio, in Germania l'uso di chip NFC sotto pelle per scopi non medici è considerato illegale, poiché viola le normative sulla privacy e la protezione dei dati personali. Questo significa che ci sono ancora molte domande senza risposta sulla sicurezza, l'efficacia e gli eventuali effetti a lungo termine di questa tecnologia. \\\\
In conclusione, l'implantologia di chip NFC sotto pelle è un argomento controverso che suscita un grande interesse e una grande preoccupazione. Mentre questa tecnologia può offrire una serie di vantaggi, ci sono anche molte preoccupazioni legate alla privacy, alla sicurezza e alla salute che devono essere prese in considerazione. Prima di decidere di utilizzare questa tecnologia, è importante valutare attentamente i rischi e i benefici e consultare un medico qualificato.

\chapter{Conclusioni}
In conclusione, l'utilizzo di NFC (Near Field Communication) è sempre più diffuso nella vita quotidiana e offre numerose applicazioni che semplificano e migliorano la nostra esperienza. Dalla gestione delle carte di credito alla fruizione di servizi in luoghi pubblici, dall'automazione degli edifici alla tracciabilità degli oggetti, NFC si è dimostrata una tecnologia versatile ed efficiente.\\\\ Con la crescente diffusione di smartphone e dispositivi con tecnologia NFC, la portata e l'utilizzo di questa tecnologia continueranno ad aumentare. Inoltre, lo sviluppo di nuove tecnologie, come l'Internet delle Cose (IoT) e la connettività 5G, apre nuove opportunità per l'utilizzo di NFC in settori come la sicurezza, la logistica e la gestione dei dati.\\\\ In sintesi, NFC è una tecnologia in continua evoluzione che offre numerose possibilità per semplificare e migliorare la nostra vita quotidiana. La sua diffusione e le sue applicazioni continueranno ad espandersi, portando a un futuro sempre più integrato e connesso.

\end{document}
